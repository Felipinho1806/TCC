\chapter{Introdução}

Conforme o passar dos anos e de diversos avanços na área tecnológica, a tecnologia digital passou por transformações significativas que mexeram com a maneira da qual as pessoas viam e interagiam com os dispositivos eletrônicos, principalmente os móveis. Os celulares se tornaram uma ferramentas indispensável no dia a dia, oferecendo diversas funcionalidades como serviços, comunicação, educação e entretenimento. Entretanto, mesmo com o aumento do acesso a essas tecnologias, a inclusão digital continua sendo um desafio importante que muitas vezes é ignorado, especialmente para pessoas com portadoras de deficiência, idosos e populações em situação de vulnerabilidade.

Inclusão digital é ligada diretamente à ao conceito de acessibilidade, pois para tornar a tecnologia acessível é preciso mais do que resolver o problema de fornecimento de dispositivos e conexão à internet, o qual já é crítico pois de acordo com reportagem da \cite{g1_acesso_pleno_internet_2022} menos de um terço da população possui acesso pleno a internet, também é preciso se preocupar com o uso de interfaces que respeitem limitações e necessidades específicas dos diversos perfis dos diversos usuários. O design de interfaces móveis deve ser pensado de forma a promover o fácil entendimento e progresso, garantindo que todos possam se beneficiar das possibilidades oferecidas pela tecnologia digital que cresce cada vez mais como é mostrado pela \cite{veja_pais_se_digitaliza_2024} que segundo uma análise recente, foi revelado que o setor de tecnologia brasileiro cresceu cerca de 86\% entre os anos de 2020 e 2024, expondo a intensa digitalização da economia.


\label{sec:introducao}

%  Exemplo de imagem
\begin{figure}[H]
\centering
\includegraphics[width=0.5\linewidth]{imagens/cefet.png}
\caption{Exemplo de foto.}
\label{fig:cefet}
\end{figure}

%  Exemplo de lista
\begin{itemize}
    \item Primeiro item
    \item Segundo item
    \item Terceiro item
\end{itemize}

\begin{itemize}
    \item Primeiro item
    \item Segundo item
    \item Terceiro item
\end{itemize}

%  Exemplo de tabela
\input{tabelas/tabela}

% Exemplo de citação
\citep{8890660} % Citação pra final de frase
\cite{8890660} % Citação para mencionar o autor
