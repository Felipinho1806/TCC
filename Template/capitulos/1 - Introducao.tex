\chapter{Introdução}

Conforme o passar dos anos e de diversos avanços na área tecnológica, a tecnologia digital passou por transformações significativas que mexeram com a maneira da qual as pessoas viam e interagiam com os dispositivos eletrônicos, principalmente os móveis. Os celulares se tornaram uma ferramentas indispensável no dia a dia, oferecendo diversas funcionalidades como serviços, comunicação, educação e entretenimento. Entretanto, mesmo com o aumento do acesso a essas tecnologias, a inclusão digital continua sendo um desafio importante que muitas vezes é ignorado, especialmente para pessoas com portadoras de deficiência, idosos e populações em situação de vulnerabilidade.

Inclusão digital é ligada diretamente à ao conceito de acessibilidade, pois para tornar a tecnologia acessível é preciso mais do que resolver o problema de fornecimento de dispositivos e conexão à internet, o qual já é crítico pois de acordo com reportagem da \cite{g1_acesso_pleno_internet_2022} menos de um terço da população possui acesso pleno a internet, também é preciso se preocupar com o uso de interfaces que respeitem limitações e necessidades específicas dos diversos perfis dos diversos usuários. O design de interfaces móveis deve ser pensado de forma a promover o fácil entendimento e progresso, garantindo que todos possam se beneficiar das possibilidades oferecidas pela tecnologia digital que cresce cada vez mais como é mostrado pela \cite{veja_pais_se_digitaliza_2024} que segundo uma análise recente, foi revelado que o setor de tecnologia brasileiro cresceu cerca de 86\% entre os anos de 2020 e 2024, expondo a intensa digitalização da economia.

%Passar pente fino dps
\subsection*{Justificativa}

O acelerado avanço da digitalização de serviços fez o ambiente digital uma parte central do dia a dia, desde serviços bancários até o acesso a serviços públicos e privados. No Brasil, por exemplo, 88,0\% das pessoas com 10 anos ou mais utilizaram a internet em 2023 (IBGE DNVVV), e enquanto isso, o uso da internet pela população idosa (60 anos ou mais) cresceu: de 44,8\% em 2019 para 66,0\% em 2023 (DE ACORDO COM O IBGE!!! https://g1.globo.com/tecnologia/noticia/2025/07/24/em-5-anos-uso-da-internet-no-brasil-acelera-e-chega-a-69percent-entre-os-idosos-diz-ibge.ghtml).

Porém mesmo que mais idosos estejam "online", muitos acabam esbarrando em barreiras técnicas, cognitivas ou de usabilidade: interfaces complexas, linguagem pouco acessível ou contrastes visuais inadequados. Para que seja possível a transformação digital dos serviços, essas barreiras devem ser levadas em contas, pois pode trazer à exclusão -- situação que acentua vulnerabilidades sociais.

Ademais, o crescimento da digitalização nos serviços diários combinam com a expectativa de autonomia do usuário no uso de aplicativos (bancos, governamentais, etc). Segundo o Instituto Brasileiro de Geografia e Estatística (IBGE), entre os não usuários da internet em 2023, 33,2\% listaram "não saber como usar" como motivo do afastamento das redes. Ainda que o acesso digital esteja aumentando cada vez mais, a preocupação com a adequação dos serviços não está evoluindo no mesmo ritmo, criando uma distância entre conectividade e usabilidade no mundo real.

% bibliograficas
% INSTITUTO BRASILEIRO DE GEOGRAFIA E ESTATÍSTICA (IBGE). Em 2023, 88,0 % das pessoas com 10 anos ou mais utilizaram a Internet. Agência de Notícias, 2024. Disponível em: https://agenciadenoticias.ibge.gov.br/agencia-noticias/2012‑news/41026‑em‑2023‑87‑2‑das‑pessoas‑com‑10‑anos‑ou‑mais‑utilizaram‑internet
% . Acesso em: 22 out. 2025. 
% Agência de Notícias - IBGE
% +1

% INSTITUTO BRASILEIRO DE GEOGRAFIA E ESTATÍSTICA (IBGE). Uso da internet no país cresce mais entre idosos. Notícias UOL, 2024. Disponível em: https://noticias.uol.com.br/ultimas‑noticias/2024/08/16/uso‑de‑internet‑no‑pais‑cresce‑mais‑entre‑idosos‑mostra‑ibge.htm
% . Acesso em: 22 out. 2025. 
% UOL Notícias

% MINISTÉRIO DAS COMUNICAÇÕES. Cresce o número de pessoas idosas com acesso à internet, segundo IBGE. 28 jul. 2025. Disponível em: https://www.gov.br/mcom/pt‑br/noticias/2025/julho/cresce‑o‑numero‑de‑pessoas‑idosas‑com‑acesso‑a‑internet‑segundo‑ibge
% . Acesso em: 22 out. 2025. 
% Serviços e Informações do Brasil

\label{sec:introducao}

%  Exemplo de imagem
\begin{figure}[H]
\centering
\includegraphics[width=0.5\linewidth]{imagens/cefet.png}
\caption{Exemplo de foto.}
\label{fig:cefet}
\end{figure}

%  Exemplo de lista
\begin{itemize}
    \item Primeiro item
    \item Segundo item
    \item Terceiro item
\end{itemize}

\begin{itemize}
    \item Primeiro item
    \item Segundo item
    \item Terceiro item
\end{itemize}

%  Exemplo de tabela
\input{tabelas/tabela}

% Exemplo de citação
\citep{8890660} % Citação pra final de frase
\cite{8890660} % Citação para mencionar o autor
